\documentclass[10pt,twoside]{article}

% Math symbols
\usepackage{amsmath}
\usepackage{amssymb}

% Headers/Footers
\usepackage{fancyhdr}

% Colors
\usepackage[usenames,dvipsnames]{color}

% Importing and manipulating graphics
\usepackage{graphicx}
\usepackage{caption}
\usepackage{lscape}

% Misc packages
\usepackage{verbatim}
\usepackage{dcolumn}
\usepackage{ifpdf}
\usepackage{enumerate}

% PDF Bookmarks and hyperref stuff
\usepackage[
  bookmarks=true,
  bookmarksnumbered=true,
  colorlinks=true,
  filecolor=blue,
  linkcolor=blue,
  urlcolor=blue,
  hyperfootnotes=true
  citecolor=blue
]{hyperref}

% Improved hyperlinking to figures
% (include after hyperref)
\usepackage[all]{hypcap}

% Improved citation handling
% (include after the hyperref stuff)
\usepackage{cite}

% Pretty-print code
\usepackage{listings}

% -----------------------------------------------------------------
% Hyper-references
% -----------------------------------------------------------------
%
% This page has lots of good advice on hyper-references, including
% the use of the hypcap package and \phantomsection for generating
% labels to text.
%
% http://en.wikibooks.org/wiki/LaTeX/Labels_and_Cross-referencing
%
% -----------------------------------------------------------------
% Setup the margins
% -----------------------------------------------------------------
% Footer Template

% Set left margin - The default is 1 inch, so the following
% command sets a 1.25-inch left margin.
\setlength{\oddsidemargin}{0.25in}
\setlength{\evensidemargin}{0.25in}

% Set width of the text - What is left will be the right
% margin. In this case, right margin is
% 8.5in - 1.25in - 6in = 1.25in.
\setlength{\textwidth}{6in}

% Set top margin - The default is 1 inch, so the following
% command sets a 0.75-inch top margin.
%\setlength{\topmargin}{-0.25in}

% Set height of the header
\setlength{\headheight}{0.3in}

% Set vertical distance between the header and the text
\setlength{\headsep}{0.2in}

% Set height of the text
\setlength{\textheight}{8.5in}

% Set vertical distance between the text and the
% bottom of footer
\setlength{\footskip}{0.4in}

% -----------------------------------------------------------------
% Allow floats to take up more space on a page.
% -----------------------------------------------------------------

% see page 142 of the Companion for this stuff and the
% documentation for the fancyhdr package
\renewcommand{\textfraction}{0.05}
\renewcommand{\topfraction}{0.95}
\renewcommand{\bottomfraction}{0.95}
% dont make this too small
\renewcommand{\floatpagefraction}{0.35}
\setcounter{totalnumber}{5}

% -----------------------------------------------------------------
% Abbreviated symbols
% -----------------------------------------------------------------
\newcommand{\sinc}{\ensuremath{\,\text{sinc}}}
\newcommand{\rect}{\ensuremath{\,\text{rect}}}

% =================================================================
% The document starts here
% =================================================================
%
\begin{document}
\title{Power Supply Capacitors}
\author{D. W. Hawkins (dwh@ovro.caltech.edu)}
\date{\today}
%\date{August 31, 2012}
\maketitle

% No header/footer on the first page
\thispagestyle{empty}

\tableofcontents

% start the intro on an odd page
\cleardoublepage
%\clearpage

% Set up the header/footer
\pagestyle{fancy}
\lhead{Power Supply Capacitors}
\chead{}
\rhead{\today}
\lfoot{}
\cfoot{}
\rfoot{\thepage}
\renewcommand{\headrulewidth}{0.4pt}
\renewcommand{\footrulewidth}{0.4pt}

% Set the listings package language to Tcl
\lstset{language=Tcl}

% =================================================================
\section{Introduction}
% =================================================================

Switch-mode power supplies (SMPS) use capacitors, inductors, 
MOSFETs, and switching controllers to convert an (unregulated)
input power source into a regulated output power source.
An SMPS circuit contains signal paths carrying pulsed
current, with fast edge-rates. These signal paths require
capacitors with high-capacitance, low equivalent series 
resistance (ESR), and low equivalent series inductance (ESL).
This document contains details on capacitors suitable for
use in power supply designs.

% =================================================================
\section{Multi-layer Ceramic Capacitors}
% =================================================================

Multi-layer capacitor capacitors (MLCC), or simply {\em ceramic}
capacitors, are excellent for use in power supplies. 
The following capacitor characteristics are critical to the
design of a power supply, however, these details are
often {\em not} included in the device data sheets;
%
\begin{itemize}
\item Equivalent series resistance (ESR)
\item Equivalent series inductance (ESL)
\item RMS current carrying capability (thermal heating)
\item The change in capacitance value with applied DC voltage

For example, a power supply with an input voltage of 12V
may use an input decoupling capacitor with a face-value of
10$\mu$F, however, under a 12V DC bias, the {\em actual}
capacitance value could be as low as 6$\mu$F. If the power
supply design required an input capacitance of {\em at least}
10$\mu$F, then parallel capacitors would be required.

\end{itemize}
%
These characteristics are not included in data sheets, since 
some of these parameters are multi-dimensional, i.e., they
change (non-linearly) with frequency and/or with applied DC voltage.

The following sections analyze capacitor characteristics
and tools from multiple vendors (in alphabetical order).


% -----------------------------------------------------------------
% Surface Mount Capacitor Footprints
% -----------------------------------------------------------------

\begin{table}
\caption{Surface mount capacitor footprint sizes.}
\label{tab:smt_footprints}
\begin{center}
\begin{tabular}{|c|c||c|c|}
\hline
\multicolumn{2}{|c||}{Imperial} & \multicolumn{2}{c|}{Metric}\\
\hline
Code & Dimensions (mils)     & Code & Dimensions (mm)\\
\hline\hline
&&&\\
0603 &  $63\times31$ & 1608M & $1.60\times0.80$\\
0805 &  $79\times49$ & 2012M & $2.00\times1.25$\\
1206 & $126\times63$ & 3216M & $3.20\times1.60$\\
1210 & $126\times98$ & 3225M & $3.20\times2.50$\\
&&&\\
\hline
\end{tabular}
\end{center}
\end{table}

\clearpage
% -----------------------------------------------------------------
\subsection{AVX}
% -----------------------------------------------------------------

Figure~\ref{fig:avx_spicap_gui} shows the AVX capacitor characterization
tool \href{http://www.avx.com/SpiApps/default.asp}{SpiCap}
(version 3.0)\footnote{The AVX web site indicates that the tool
SpiCALCI 7.0 can also be used for power supply capacitor selection,
however, the tool currently only works for AC-DC converter designs.
For example, if you select a 1210 package size and X7R dielectric,
the tool generates a warning message.}. The tool is incomplete;
there is no option for entering part numbers, no details on
RMS current handling or thermal conductivity are provided,
and there is no option to generate a SPICE netlist (you have to
make a note of the calculated parameters and create your own netlist).

Table~\ref{tab:avx_caps} shows a selection of AVX ceramic capacitor
part numbers obtained from \href{http://www.digikey.com}{Digikey}.
SpiCap was used to generate estimates of ESR and ESL. Various
package sizes were selected to show the change in ESL.
Constructing the table showed another limitation of SpiCap;
the characteristics of valid part numbers are not always
supported (i.e., the missing ESR values).

The AVX~\href{http://www.avx.com/docs/masterpubs/smccp.pdf}
{Surface Mount Ceramic Capacitor Products} catalog  (version 12.4)
contains details of AVX capacitor products, eg., 
see p18 for X7R dielectric, and p25 for X5R. There are no
details on capacitance change versus DC bias for the X7R and
X5R dielectrics for specific parts (p28 has a plot for Y5V dielectric).
There is a general description of MLCC design on p99, and that
description has comments on the change in capacitance
with applied AC and DC voltages, and change in temperature.
The AVX document refers the reader to use SpiCap for component
characteristics.

The main \textcolor{red}{problem} with the AVX documentation is that
there are no details on the {\em power} handling 
capability of the devices, or the thermal conductivity of the packages.

\clearpage
% -----------------------------------------------------------------
% SpiCap3.0
% -----------------------------------------------------------------
%
\begin{figure}[p]
  \begin{center}
    \includegraphics[width=\textwidth]{figures/avx_spicap_gui.png}\\
  \end{center}
  \caption{AVX SpiCap 3.0 GUI.
  The GUI shows the characteristics of a 10$\mu$F capacitor
  (25V rating, X5R dielectric, 1210 package) with a 12V DC bias
  (48\% of rated voltage). The capacitance value reduces to
  8.88$\mu$F due to the DC bias. The ESR at 500kHz is 4m$\Omega$.
  The 1210 package ESL is 900pH.}
  \label{fig:avx_spicap_gui}
\end{figure}

% -----------------------------------------------------------------
% AVX part numbers
% -----------------------------------------------------------------

\begin{table}[p]
\caption{Example AVX ceramic capacitor characteristics.}
\label{tab:avx_caps}
\begin{center}
\begin{tabular}{|c|c|c|c|c|c|c|}
\hline
Part   & Package & Voltage & Dielectric & Capacitance  & ESR       & ESL\\
Number &         & Rating  &            & (0V DC bias) & (@500kHz) & \\
\hline\hline
&&&&&&\\
06033D105KAT2A & 0603 & 25V & X5R & 1$\mu$F &             & 500pH\\
&&&&&&\\
08053D105KAT2A & 0805 & 25V & X5R & 1$\mu$F & 13m$\Omega$ & 600pH\\
08053C105KAT2A & 0805 & 25V & X7R & 1$\mu$F &             & 600pH\\
&&&&&&\\
12063D105KAT2A & 1206 & 25V & X5R &   1$\mu$F & 15m$\Omega$ & 1000pH\\
12063C105KAT2A & 1206 & 25V & X7R &   1$\mu$F & 12m$\Omega$ & 1000pH\\
12063D475KAT2A & 1206 & 25V & X5R & 4.7$\mu$F &  5m$\Omega$ & 1000pH\\
12063D106KAT2A & 1206 & 25V & X5R &  10$\mu$F &             & 1000pH\\
&&&&&&\\
12103D105KAT2A & 1210 & 25V & X5R &   1$\mu$F &             & 900pH\\
12103C105KAT2A & 1210 & 25V & X7R &   1$\mu$F & 12m$\Omega$ & 900pH\\
12103D476KAT2A & 1210 & 25V & X5R & 4.7$\mu$F &  5m$\Omega$ & 900pH\\
12103D106KAT2A & 1210 & 25V & X5R &  10$\mu$F &  3m$\Omega$ & 900pH\\
12103C106KAT2A & 1210 & 25V & X7R &  10$\mu$F &  3m$\Omega$ & 900pH\\
&&&&&&\\
\hline
\end{tabular}
\end{center}
\end{table}

\clearpage
% -----------------------------------------------------------------
\subsection{Kemet}
% -----------------------------------------------------------------

Figures~\ref{fig:kemet_cap_selector} through~\ref{fig:kemet_power}
show screen-shots of the Kemet component characterization tool
\href{http://www.kemet.com/kemet/web/homepage/kechome.nsf/weben/kemsoft}
{KEMET Spice} (version 3.9.56). 
Figure~\ref{fig:kemet_cap_selector} shows the capacitor selection
GUI, with the 10$\mu$F C1210C106K3PAC capacitor selected.
Figure~\ref{fig:kemet_Z_and_ESR} shows the capacitor impedance and 
ESR\footnote{Menu option: Change Plot$\rightarrow$Plot Z and R.},
Figure~\ref{fig:kemet_C_and_ESL} shows the capacitance value and 
ESL\footnote{Menu option: Change Plot$\rightarrow$Plot Cap and L.},
while Figure~\ref{fig:kemet_power} shows the power 
handling\footnote{Menu option: Change Plot$\rightarrow$Plot Current(I).}.
%
Figures~\ref{fig:kemet_Z_and_ESR} through Figure~\ref{fig:kemet_power}
show the capacitor characteristics for a DC bias of 0V and 
12V\footnote{Menu option: Bias Change$\rightarrow$Custom, and
then enter 12V.}.

KEMET Spice can generate a capacitor SPICE model (netlist) for use in other
SPICE simulators. Figure~\ref{fig:kemet_model} shows the SPICE model
generator display. The SPICE model is generated based on a part 
number, DC bias voltage, and frequency. The frequency can be selected
by moving the mouse on the screen, press CTRL+0 to lock the screen,
and then selecting the model generator\footnote{Menu option:
MonitorXY$\rightarrow$Model Display.}. The SPICE model is
generated by clicking on the {\em Output List} button in
Figure~\ref{fig:kemet_model}, selecting an output file name, and
the clicking {\em Save List}.

Table~\ref{tab:kemet_caps} shows a selection of KEMET ceramic capacitor
part numbers obtained from \href{http://www.digikey.com}{Digikey}.
The part numbers in the table include both $\pm10\%$ and $\pm20\%$
tolerance parts. KEMET Spice does not have an option for selecting 
$\pm20\%$ tolerance parts so the parameters for $\pm10\%$ parts were used.
KEMET SPICE was used to estimate capacitance, ESR, ESL, and power handling.

The capacitor power handling capability is determined based on
the power dissipation that causes a temperature rise of 20$^\circ$C.
For example, in Figure~\ref{fig:kemet_power} the allowable power dissipation
$\Delta P = \Delta T/\theta_{\rm JA} = 20/55 = 363.64$mW, where
$\theta_{\rm JA} = 55^\circ$C/W is the thermal conductivity.
The RMS current in Figure~\ref{fig:kemet_power} is calculated using
this power limit along with the ESR shown in Figure~\ref{fig:kemet_Z_and_ESR}.
The RMS current in Figure~\ref{fig:kemet_power}(a) is calculated as
$I = \sqrt{P/\text{ESR}} = \sqrt{363.64\text{mW}/5.87\text{m}\Omega} = 7.87$A,
and in Figure~\ref{fig:kemet_power}(b) is calculated as
$I = \sqrt{363.64\text{mW}/4.97\text{m}\Omega} = 8.55$A.

Table~\ref{tab:kemet_caps} shows how the capacitance value and ESR change
with applied DC bias voltage (the change in current is due to the change in ESR).
SMPS power supplies always use capacitors with a non-zero DC bias, so
the capacitance value will {\em always} be lower than the nominal
capacitance value. If a power supply design requires a specific value
of capacitance, then higher valued capacitors must be used. A non-zero
DC bias lowers the ESR, resulting in higher current handling.

The KEMET Spice interface has several minor bugs;
%
\begin{itemize}
\item If you are using a DC bias other than the default value of
0V, then the DC bias value needs to be re-selected when you change
parts (via File$\rightarrow$Choose Another Ceramic), otherwise
the curves and values displayed on the plots are wrong
(they are calculated using a 0V DC bias, regardless of the
Change Bias menu selection).

When changing parts, unlock the screen (using CTRL+0), re-select the
desired DC bias, move the cursor to the desired frequency, 
relock the screen, and then view the capacitor characteristics
using the Change Plot menu.
%
\item When the screen is locked, eg., to obtain parameters at 500kHz,
and you select a different display, the frequency will sometimes jump slightly
higher, requiring you to unlock the screen, move the cursor, and relock.
\end{itemize}
%
Minor bugs aside, the KEMET Spice tool is very useful, as it provides
all of the capacitor characteristics required when designing power supplies.

% -----------------------------------------------------------------
% SpiCap3.0
% -----------------------------------------------------------------
%
\begin{figure}[p]
  \begin{center}
    \includegraphics[width=\textwidth]{figures/kemet_cap_selector.png}\\
  \end{center}
  \caption{KEMET Spice capacitor selection GUI.}
  \label{fig:kemet_cap_selector}
\end{figure}

\begin{figure}[p]
  \begin{center}
    \includegraphics[width=\textwidth]{figures/kemet_C1210C106K3PAC_Z_and_ESR_DC0V.png}\\
    (a) 0V DC bias\\
    $\quad$\\
    \includegraphics[width=\textwidth]{figures/kemet_C1210C106K3PAC_Z_and_ESR_DC12V.png}\\
    (b) 12V DC bias\\
  \end{center}
  \caption{KEMET 10$\mu$F C1210C106K3PAC capacitor impedance and ESR.}
  \label{fig:kemet_Z_and_ESR}
\end{figure}

\begin{figure}[p]
  \begin{center}
    \includegraphics[width=\textwidth]{figures/kemet_C1210C106K3PAC_C_and_ESL_DC0V.png}\\
    (a) 0V DC bias\\
    $\quad$\\
    \includegraphics[width=\textwidth]{figures/kemet_C1210C106K3PAC_C_and_ESL_DC12V.png}\\
    (b) 12V DC bias\\
  \end{center}
  \caption{KEMET 10$\mu$F C1210C106K3PAC capacitor capacitance value and ESL.}
  \label{fig:kemet_C_and_ESL}
\end{figure}

\begin{figure}[p]
  \begin{center}
    \includegraphics[width=\textwidth]{figures/kemet_C1210C106K3PAC_power_DC0V.png}\\
    (a) 0V DC bias\\
    $\quad$\\
    \includegraphics[width=\textwidth]{figures/kemet_C1210C106K3PAC_power_DC12V.png}\\
    (b) 12V DC bias\\
  \end{center}
  \caption{KEMET 10$\mu$F C1210C106K3PAC capacitor power handling.}
  \label{fig:kemet_power}
\end{figure}

\begin{figure}[p]
  \begin{center}
    \includegraphics[width=\textwidth]{figures/kemet_C1210C106K3PAC_model.png}\\
  \end{center}
  \caption{KEMET 10$\mu$F C1210C106K3PAC capacitor model.}
  \label{fig:kemet_model}
\end{figure}

% -----------------------------------------------------------------
% KEMET part numbers
% -----------------------------------------------------------------

\begin{landscape}
\begin{table}[p]
\small
\caption{Example KEMET ceramic capacitor characteristics (@500kHz).}
\label{tab:kemet_caps}
\begin{center}
\begin{tabular}{|c|c|c|c|c|c||c|c|c||c|c|c||c|c|}
\hline
Part   & Cap. & Voltage & Tol. & Dielectric & Package &
\multicolumn{3}{c||}{DC Bias of 0V}  &
\multicolumn{3}{c||}{DC Bias of 50\%} & ESL & Thermal\\
%
\cline{7-12}
Number &   & Rating  &           &            &         &
Cap. & ESR & Current    &
Cap. & ESR & Current    &       & ($\theta_{\rm JA}$)\\
%
       & ($\mu$F) & (V) & (\%) & & & 
($\mu$F) & (m$\Omega$) & (A$_{\rm RMS}$) &
($\mu$F) & (m$\Omega$) & (A$_{\rm RMS}$) & (pH) & ($^\circ$C/W)\\
\hline\hline
&&&&&&&&&&&&&\\
C0603C105K3PAC &   1 &  25 & $\pm10$ & X5R & 0603 &  0.92 & 19 &  4.9 & 0.59 & 13 &  5.9 & 1250 & 45\\
C0603C225K4PAC & 2.2 &  16 & $\pm10$ & X5R & 0603 &  2.0  &  9 &  7.8 & 1.4  &  6 &  9.2 & 1250 & 38\\
C0603C106M9PAC &  10 & 6.3 & $\pm20$ & X5R & 0603 &  9.2  &  5 & 10.5 & 5.8  &  3 & 13.0 &  830 & 41\\
&&&&&&&&&&&&&\\
C0805C105K3PAC &   1 &  25 & $\pm10$ & X5R & 0805 &  0.92 &  9 &  5.3 & 0.59 & 6 &  6.5 & 1110 & 78\\
C0805C475K3PAC & 4.7 &  25 & $\pm10$ & X5R & 0805 &  4.3  &  4 &  9.6 & 2.8  & 3 & 11.8 & 1180 & 55\\
C0805C106K4PAC &  10 &  16 & $\pm10$ & X5R & 0805 &  9.2  &  4 & 10.8 & 7.9  & 3 & 11.6 & 1000 & 45\\
C0805C476M9PAC &  47 & 6.3 & $\pm20$ & X5R & 0805 & 42.5  &  2 & 17.1 & 16.2 & 1 & 27.1 &  500 & 35\\
&&&&&&&&&&&&&\\
C1206C105K3PAC &   1 &  25 & $\pm10$ & X5R & 1206 &  0.92 & 21 &  3.3 & 0.59 & 14 &  4.1 &  950 & 85\\
C1206C106K3PAC &  10 &  25 & $\pm10$ & X5R & 1206 &  9.2  &  7 &  7.0 & 5.9  &  5 &  8.7 & 1180 & 55\\
C1206C107M9PAC & 100 & 6.3 & $\pm20$ & X5R & 1206 & 90.5  &  2 & 18.4 & 36.7 &  1 & 28.7 & 1060 & 28\\
&&&&&&&&&&&&&\\
C1210C105K3PAC &   1 &  25 & $\pm10$ & X5R & 1210 &  0.92 & 19 &  3.3 &  0.8 & 17 &  3.5 &  730 & 95\\
C1210C106K3PAC &  10 &  25 & $\pm10$ & X5R & 1210 &  9.2  &  6 &  7.8 &  5.9 &  4 &  9.8 &  990 & 55\\
C1210C107M9PAC & 100 & 6.3 & $\pm20$ & X5R & 1210 & 90.4  &  3 & 10.5 & 57.4 &  2 & 13.2 &  850 & 60\\
&&&&&&&&&&&&&\\
\hline
\end{tabular}
\end{center}
\end{table}
\end{landscape}

\clearpage
% -----------------------------------------------------------------
\subsection{Murata}
% -----------------------------------------------------------------

Figures~\ref{fig:murata_simsurfing_cap_selector} 
through~\ref{fig:murata_ESL} show screen-shots of the Murata 
component characterization tool
\href{http://www.murata.com/simsurfing/}{SimSurfing}. 
%
Figure~\ref{fig:murata_simsurfing_cap_selector} shows the capacitor
selection GUI, with the 10$\mu$F GRM32DR61E106KA12 capacitor selected.
Figure~\ref{fig:murata_DC_bias} shows the capacitance value change
with applied DC bias (C-DC Bias GUI button).
%
Figure~\ref{fig:murata_power} shows the change in temperature
versus RMS current, i.e., the power handling capability of the
capacitor (Temp. Rise GUI button).
%
Figures~\ref{fig:murata_Z}, \ref{fig:murata_C}, \ref{fig:murata_ESR}, 
and~\ref{fig:murata_ESL} show the capacitor impedance, capacitance,
ESR, and ESL versus frequency for a DC bias of 0V and 12.5V (50\% of 
rated voltage). These plots were generated using the $|$Z$|$, C, R, and L
GUI buttons respectively. Each plot contains curves generated 
using two DC bias settings (the GUI check-box Display Multiple Graphs
need to be checked to have the plots overlay).

Table~\ref{tab:murata_caps} shows a selection of Murata ceramic capacitor
part numbers obtained from \href{http://www.digikey.com}{Digikey}.
The SimSurfing tool was used to determine the parameters shown in
the table. The SimSurfing measurement guide (accessed via the help menu),
indicates that the temperature response is calculated using a DC
bias of 50\% of the voltage rating, i.e., the temperature response
plot provides the RMS current given under the {\em DC Bias of 50\%}
column in the table. The RMS current for the {\em DC Bias of 0V}
column was calculated by scaling the power by the ESR, i.e.,
%
\begin{equation}
I_{\rm DC0V} = I_{\rm DC50\%}\sqrt{\frac{R_{\rm DC50\%}}{R_{\rm DC0V}}},
\end{equation}
% 
where the subscripts on $I$ and $R$ indicate the DC bias. 
SimSurfing does not provide details on the thermal
resistance of the capacitors, so it was estimated via;
%
\begin{equation}
\theta_{\rm JA} = \frac{20^\circ\text{C}}{I_{\rm DC50\%}^2R_{\rm DC50\%}},
\end{equation}
%
where $I_{\rm DC50\%}$ is the RMS current at a DC bias of 50\% of
the rated voltage read from the temperature rise plot at $20^\circ\text{C}$.
The SimSurfing measurement guide recommends that the capacitor temperature
rise is under $20^\circ\text{C}$.

Table~\ref{tab:murata_caps} demonstrates a potential issue with the
use of ceramic capacitors in power supply designs; at a DC bias of
50\%, the 4.7$\mu$F and 10$\mu$F 0805 capacitors have virtually the
same capacitance of 1.8$\mu$F and 1.9$\mu$F respectively!
Figure~\ref{fig:murata_10uF_cap_change} plots the change in capacitance
for the 10$\mu$F part in 1210, 1206, and 0603 packages.
The 1210 and 1206 packages are virtually identical, yet the
1206 package suffers an {\em additional} 30\% capacitance
loss relative to the 1210 part (which has already lost 30\%)!
This loss in capacitance {\em must} be accounted for when determining
the number of capacitors required to meet a specific total capacitance
requirement.

% -----------------------------------------------------------------
% SimSurfing
% -----------------------------------------------------------------
%
\begin{figure}[p]
  \begin{center}
    \includegraphics[width=\textwidth]{figures/murata_simsurfing_gui.png}\\
  \end{center}
  \caption{Murata SimSurfing capacitor selection GUI.}
  \label{fig:murata_simsurfing_cap_selector}
\end{figure}

% -----------------------------------------------------------------
% Screen captures
% -----------------------------------------------------------------
%
% * Scale Firefox to desired size
% * Shift-Ctrl-Alt-PrScr
% * Paste into Corel PhotoPaint
% * Crop the graph area of the display (eg., 1027 x 670)
% * Save as PNG
% * Convert to BMP and back to PNG (to stop pdflatex from crashing)
%
\begin{figure}[p]
  \begin{center}
    \includegraphics[width=0.9\textwidth]{figures/murata_GRM32DR61E106KA12_DC_bias.png}
  \end{center}
  \caption{Murata 10$\mu$F GRM32DR61E106KA12 capacitor change with applied DC bias.}
  \label{fig:murata_DC_bias}
\end{figure}

\begin{figure}[p]
  \begin{center}
    \includegraphics[width=0.9\textwidth]{figures/murata_GRM32DR61E106KA12_power.png}
  \end{center}
  \caption{Murata 10$\mu$F GRM32DR61E106KA12 capacitor change in temperature with RMS current.}
  \label{fig:murata_power}
\end{figure}

\begin{figure}[p]
  \begin{center}
    \includegraphics[width=0.9\textwidth]{figures/murata_GRM32DR61E106KA12_Z.png}
  \end{center}
  \caption{Murata 10$\mu$F GRM32DR61E106KA12 capacitor impedance vs frequency for 0V DC bias and 12.5V (50\%) DC bias.}
  \label{fig:murata_Z}
\end{figure}

\begin{figure}[p]
  \begin{center}
    \includegraphics[width=0.9\textwidth]{figures/murata_GRM32DR61E106KA12_C.png}
  \end{center}
  \caption{Murata 10$\mu$F GRM32DR61E106KA12 capacitor capacitance vs frequency for 0V DC bias and 12.5V (50\%) DC bias.}
  \label{fig:murata_C}
\end{figure}

\begin{figure}[p]
  \begin{center}
    \includegraphics[width=0.9\textwidth]{figures/murata_GRM32DR61E106KA12_ESR.png}
  \end{center}
  \caption{Murata 10$\mu$F GRM32DR61E106KA12 capacitor ESR vs frequency for 0V DC bias and 12.5V (50\%) DC bias.}
  \label{fig:murata_ESR}
\end{figure}

\begin{figure}[p]
  \begin{center}
    \includegraphics[width=0.9\textwidth]{figures/murata_GRM32DR61E106KA12_ESL.png}
  \end{center}
  \caption{Murata 10$\mu$F GRM32DR61E106KA12 capacitor ESL vs frequency for 0V DC bias and 12.5V (50\%) DC bias.}
  \label{fig:murata_ESL}
\end{figure}

% -----------------------------------------------------------------
% Murata part numbers
% -----------------------------------------------------------------

\begin{landscape}
\begin{table}[p]
\small
\caption{Example Murata ceramic capacitor characteristics (@500kHz).}
\label{tab:murata_caps}
\begin{center}
\begin{tabular}{|c|c|c|c|c|c||c|c|c||c|c|c||c|c|}
\hline
Part   & Cap. & Voltage & Tol. & Dielectric & Package &
\multicolumn{3}{c||}{DC Bias of 0V}  &
\multicolumn{3}{c||}{DC Bias of 50\%} & ESL & Thermal\\
%
\cline{7-12}
Number &   & Rating  &           &            &         &
Cap. & ESR & Current    &
Cap. & ESR & Current    &       & ($\theta_{\rm JA}$)\\
%
       & ($\mu$F) & (V) & (\%) & & & 
($\mu$F) & (m$\Omega$) & (A$_{\rm RMS}$) &
($\mu$F) & (m$\Omega$) & (A$_{\rm RMS}$) & (pH) & ($^\circ$C/W)\\
\hline\hline
&&&&&&&&&&&&&\\
GRM188R61E105KA12D &    1 &  25 & $\pm10$ & X5R & 0603 &   1.0 & 14 & 2.1 & 0.43 & 19 & 1.8 & 300 & 325\\
GRM188R60J106ME47D &   10 & 6.3 & $\pm20$ & X5R & 0603 &  10.0 &  7 & 4.3 & 4.4  &  9 & 3.8 & 240 & 154\\
GRM188R60G226MEA0D &   22 &   4 & $\pm20$ & X5R & 0603 &  20.6 &  3 & 4.2 & 12.0 &  3 & 4.2 & 240 & 378\\
&&&&&&&&&&&&&\\
GRM216R61E105KA12D &    1 &  25 & $\pm10$ & X5R & 0805 &   1.0 & 24 & 2.1 & 0.44 & 29 & 1.9 & 320 & 191\\
GRM21BR61E475KA12L &  4.7 &  25 & $\pm10$ & X5R & 0805 &   5.0 &  7 & 3.4 & 1.8  &  7 & 3.4 & 280 & 247\\
GRM21BR61E106KA73L &   10 &  25 & $\pm10$ & X5R & 0805 &  10.2 &  5 & 4.2 & 1.9  &  5 & 4.2 & 280 & 227\\
GRM21BR60J226ME39L &   22 & 6.3 & $\pm20$ & X5R & 0805 &  22.0 &  4 & 4.7 & 11.9 &  4 & 4.7 & 260 & 226\\
&&&&&&&&&&&&&\\
GRM319R61E105KC36D &    1 &  25 & $\pm10$ & X5R & 1206 &   1.0 & 11 & 2.4 & 0.85 & 11 & 2.4 & 580 & 316\\
GRM31CR61E106KA12L &   10 &  25 & $\pm10$ & X5R & 1206 &  10.4 &  5 & 3.9 & 4.1  &  5 & 3.9 & 520 & 263\\
GRM31CR60J107ME39L &  100 & 6.3 & $\pm20$ & X5R & 1206 & 100.3 &  3 & 4.1 & 39.5 &  2 & 5.1 & 360 & 385\\
&&&&&&&&&&&&&\\
GRM32RR71H105KA01L &    1 &  50 & $\pm10$ & X7R & 1210 &   1.0 &  6 & 3.6 & 0.88 &  7 & 3.3 & 330 & 262\\
GRM32DR61E106KA12L &   10 &  25 & $\pm10$ & X5R & 1210 &  10.3 &  4 & 5.7 & 7.3  &  5 & 5.1 & 340 & 154\\
GRM32ER60J107ME20L &  100 & 6.3 & $\pm20$ & X5R & 1210 &  95.6 &  3 & 5.5 & 68.0 &  3 & 5.5 & 310 & 220\\
&&&&&&&&&&&&&\\
\hline
\end{tabular}
\end{center}
\end{table}
\end{landscape}

% -----------------------------------------------------------------
% Murata Capacitance change with package size
% -----------------------------------------------------------------
%
\begin{figure}[p]
  \begin{center}
    \includegraphics[width=0.9\textwidth]{figures/murata_10uF_cap_change_a.png}\\
    (a) change in capacitance\\
    $\quad$\\
    \includegraphics[width=0.9\textwidth]{figures/murata_10uF_cap_change_b.png}\\
    (b) relative change\\
  \end{center}
  \caption{Murata 10$\mu$F capacitor capacitance change with  package size.
  For a 25V voltage rated capacitor operating at a DC bias of 50\% (12.5V),
  the capacitance loss for the 1210, 1206, and 0805 packages is
  -29\%, -61\%, and -81\%  respectively. The capacitance loss in the
  1206 and 0805 packages is {\em significant}.}
  \label{fig:murata_10uF_cap_change}
\end{figure}

\clearpage
% -----------------------------------------------------------------
\subsection{Taiyo Yuden}
% -----------------------------------------------------------------

Figures~\ref{fig:taiyo_yuden_simsurfing_cap_selector} 
and~\ref{fig:taiyo_yuden_TMK325BJ106MM_characteristics} show 
screen-shots of the Taiyo Yuden 
\href{http://www.yuden.co.jp/ut/product/support/tool/}
{\em Components Selection Guide \& Data Library}.
%
Figure~\ref{fig:taiyo_yuden_simsurfing_cap_selector} shows the capacitor
selection GUI, with the 10$\mu$F TMK325BJ106MM capacitor selected.
Figure~\ref{fig:taiyo_yuden_TMK325BJ106MM_characteristics} shows 
the screen that appears when you click the {\em Plot} button
next to a capacitor part number.
%
Each of the six characteristic curves in 
Figure~\ref{fig:taiyo_yuden_TMK325BJ106MM_characteristics}
can be clicked on, and data points read from the curves.
The plots can also be saved as .bmp, .jpg, or .pdf.
Figures~\ref{fig:taiyo_yuden_Z_and_ESR} 
through~\ref{fig:taiyo_yuden_temperature} show the curves
for the 10$\mu$F TMK325BJ106MM capacitor.
%
These parameter curves are also available on the Taiyo Yuden
web page for the
\href{http://www.yuden.co.jp/ut/product/category/capacitor/TMK325BJ106MM-T.html}
{TMK325BJ106MM-T} in the 
\href{http://www.yuden.co.jp/productdata/sheet/tmk325bj106mm.pdf}
{Characteristics} data sheet.
Component characteristic curves can also be generated online in a
web browser using 
\href{http://ds.yuden.co.jp/TYCOMPAS/ut/searcherMain.do}
{TY-COMPAS} (Taiyo Yuden COMPonent Assist System).

The Taiyo Yuden tool has a couple of limitations;
%
\begin{itemize}
\item 
There is no option for selecting a new DC bias, so although
you can determine the change in capacitance value, there is
no way to determine the change in ESR.
%
\item There are no details on the DC bias used to generate the
power handling plot in Figure~\ref{fig:taiyo_yuden_power}.
A DC bias of 0V is assumed.
%
\end{itemize}
%
A SPICE model for the capacitor at a 0V DC Bias can be saved
by clicking on the {\em Save Spice Model} button on the left
side of the GUI in
Figure~\ref{fig:taiyo_yuden_TMK325BJ106MM_characteristics}.
The model is a series RLC circuit.

Table~\ref{tab:taiyo_yuden_caps} shows a selection of Taiyo Yuden
ceramic capacitor part numbers obtained 
from~\href{http://www.digikey.com}{Digikey}.
Note that the Digikey page for each part links to the
\href{http://www.yuden.co.jp/ut/product/category/capacitor/TMK325BJ106MM-T.pdf}
{Specification} PDF on the Taiyo Yuden web site. The
Specification contains only the nominal characteristics of
the capacitor, rather than the real-world details shown in the 
\href{http://www.yuden.co.jp/productdata/sheet/tmk325bj106mm.pdf}
{Characteristics} data sheet (or in Figure~\ref{fig:taiyo_yuden_TMK325BJ106MM_characteristics}). 
%
The characteristics in Table~\ref{tab:taiyo_yuden_caps} were estimated
from the characteristic curves like those in 
Figures~\ref{fig:taiyo_yuden_Z_and_ESR} 
through~\ref{fig:taiyo_yuden_temperature}.
The SPICE model RLC values are slightly different than the estimates
(perhaps due to best-fit of the SPICE model, or a different 
measurement frequency for the model parameters).
Taiyo Yuden does not provide package thermal resistance, so it
was estimated using the method similar to that used with the
Murata capacitors, but with a power estimate at 0V DC bias.

% -----------------------------------------------------------------
% Component Selection Guide and Data Library
% -----------------------------------------------------------------
%
\begin{landscape}
\begin{figure}[p]
  \begin{center}
    \includegraphics[width=210mm]{figures/taiyo_yuden_cap_selector.png}\\
  \end{center}
  \caption{Taiyo Yuden capacitor selection GUI.}
  \label{fig:taiyo_yuden_simsurfing_cap_selector}
\end{figure}
\end{landscape}

% -----------------------------------------------------------------
% Component Selection Guide and Data Library
% -----------------------------------------------------------------
%
\begin{landscape}
\begin{figure}[p]
  \begin{center}
    \includegraphics[width=210mm]{figures/taiyo_yuden_TMK325BJ106MM_all_plots.png}\\
  \end{center}
  \caption{Taiyo Yuden TMK325BJ106MM capacitor characteristics.}
  \label{fig:taiyo_yuden_TMK325BJ106MM_characteristics}
\end{figure}
\end{landscape}

% -----------------------------------------------------------------
% TMK325BJ106MM plots
% -----------------------------------------------------------------
%
% * Select the plot in the GUI
% * Select 'Save Chart'
% * Use the pull-down to save as .PDF
%
% The PDF can be imported by Inkscape if needed.
%
\begin{figure}[p]
  \begin{center}
    \includegraphics[width=\textwidth]{figures/taiyo_yuden_TMK325BJ106MM_Z_and_ESR.pdf}
  \end{center}
  \caption{Taiyo Yuden 10$\mu$F TMK325BJ106MM capacitor impedance and ESR vs frequency.}
  \label{fig:taiyo_yuden_Z_and_ESR}
\end{figure}

\begin{figure}[p]
  \begin{center}
    \includegraphics[width=\textwidth]{figures/taiyo_yuden_TMK325BJ106MM_C.pdf}
  \end{center}
  \caption{Taiyo Yuden 10$\mu$F TMK325BJ106MM capacitor capacitance vs frequency.}
  \label{fig:taiyo_yuden_C}
\end{figure}

\begin{figure}[p]
  \begin{center}
    \includegraphics[width=\textwidth]{figures/taiyo_yuden_TMK325BJ106MM_ESL.pdf}
  \end{center}
  \caption{Taiyo Yuden 10$\mu$F TMK325BJ106MM capacitor ESL vs frequency.}
  \label{fig:taiyo_yuden_ESL}
\end{figure}

\begin{figure}[p]
  \begin{center}
    \includegraphics[width=\textwidth]{figures/taiyo_yuden_TMK325BJ106MM_power.pdf}
  \end{center}
  \caption{Taiyo Yuden 10$\mu$F TMK325BJ106MM capacitor change in temperature with RMS current.}
  \label{fig:taiyo_yuden_power}
\end{figure}

\begin{figure}[p]
  \begin{center}
    \includegraphics[width=\textwidth]{figures/taiyo_yuden_TMK325BJ106MM_DC_bias.pdf}
  \end{center}
  \caption{Taiyo Yuden 10$\mu$F TMK325BJ106MM capacitor capacitance change with applied DC bias.}
  \label{fig:taiyo_yuden_DC_bias}
\end{figure}

\begin{figure}[p]
  \begin{center}
    \includegraphics[width=\textwidth]{figures/taiyo_yuden_TMK325BJ106MM_temperature.pdf}
  \end{center}
  \caption{Taiyo Yuden 10$\mu$F TMK325BJ106MM capacitor capacitance change vs temperature.}
  \label{fig:taiyo_yuden_temperature}
\end{figure}

% -----------------------------------------------------------------
% Taiyo Yuden part numbers
% -----------------------------------------------------------------

\begin{landscape}
\begin{table}[p]
\small
\caption{Example Taiyo Yuden ceramic capacitor characteristics (@500kHz).}
\label{tab:taiyo_yuden_caps}
\begin{center}
\begin{tabular}{|c|c|c|c|c|c||c|c|c|c||c||c|}
\hline
Part   & Cap. & Voltage & Tol. & Dielectric & Package &
\multicolumn{4}{c||}{DC Bias of 0V}  &
DC Bias of 50\% & Thermal\\
%
\cline{7-11}
Number &   &     & Rating     &            &         &
Cap. & ESR & ESL & Current    &
Cap. (change) & ($\theta_{\rm JA}$)\\
%
       & ($\mu$F) & (V) & (\%) & & & 
($\mu$F) & (m$\Omega$) & (pH) & (A$_{\rm RMS}$) &
($\mu$F) (\%) &($^\circ$C/W)\\
\hline\hline
&&&&&&&&&&&\\
TMK107BJ105KA &    1 & 25  & $\pm10$ & X5R & 0603 & 0.91 & 9.0 & 520 & 1.4 & 0.50 (-45.0) & 1130\\
TMK107ABJ225KA&  2.2 & 25  & $\pm10$ & X5R & 0603 & 1.7  & 8.9 & 520 & 1.8 & 0.41 (-76.0) &  690\\
JMK107BJ106MA &   10 & 6.3 & $\pm20$ & X5R & 0603 & 8.5  & 5.0 & 480 & 3.4 & 3.9  (-54.2) &  350\\
&&&&&&&&&&&\\
TMK212BJ105KG &  1.0 &  25 & $\pm10$ & X5R & 0805 & 0.88 & 10.0& 640 & 1.6 & 0.64 (-27.4) &  780\\
TMK212BBJ475KD&  4.7 &  25 & $\pm10$ & X5R & 0805 & 3.8  & 5.8 & 570 & 2.5 & 0.79 (-79.2) &  550\\
TMK212BBJ106KG& 10.0 &  25 & $\pm20$ & X5R & 0805 & 8.0  & 3.1 & 530 & 2.7 & 1.8  (-78.0) &  880\\
&&&&&&&&&&&\\
TMK316BJ105KD &    1 &  25 & $\pm10$ & X5R & 1206 & 0.90 & 15.8& 870 & 1.4 & 0.63 (-29.7) &  650\\
TMK316BJ475KD &  4.7 &  25 & $\pm10$ & X5R & 1206 & 4.3  & 5.1 & 700 & 2.4 & 1.8  (-58.5) &  680\\
TMK316BJ106KL &   10 &  25 & $\pm10$ & X5R & 1206 & 9.0  & 3.0 & 690 & 3.4 & 4.0  (-55.3) &  580\\
&&&&&&&&&&&\\
HMK325B7105KN &    1 & 100 & $\pm10$ & X5R & 1210 & 0.95 & 9.3 & 460 & 1.6 & 0.53 (-44.1) &  840\\
TMK325BJ106MM &   10 &  25 & $\pm10$ & X5R & 1210 & 9.2  & 3.5 & 500 & 3.5 & 6.4  (-34.2) &  470\\
EMK325BJ476MM &   47 &  16 & $\pm10$ & X5R & 1210 & 41.5 & 2.9 & 460 & 3.1 & 19.5 (-53.0) &  720\\
&&&&&&&&&&&\\
\hline
\end{tabular}
\end{center}
\end{table}
\end{landscape}

\clearpage
% -----------------------------------------------------------------
\subsection{TDK}
% -----------------------------------------------------------------

The TDK \href{http://www.tdk.com/design-tools.php}
{Technical Support Tools} web page provides several tools for
component selection and characterization.
There are two component modeling tools; 
\href{http://www.tdk.com/seat.php}
{Selection Assistant of TDK components} (SEAT) and 
\href{http://www.tdk.co.jp/ccv/index.asp}
{Component Characteristics Viewer} (CCV). 
%
A comparison of the tools can be found in the application note
\href{http://www.tdk.com/pdf/SEATs_vs_CCV.pdf}{SEAT vs CCV}.
The main difference between the tools is that CCV is an online tool,
whereas SEAT is run locally.
%
SPICE models for components are available from the
\href{http://www.tdk.com/tvcl.php}{TDK Virtual Component Library} (TVCL).


Figures~\ref{fig:tdk_seat_gui_Z_and_ESR} through~\ref{fig:tdk_seat_gui_power}
show screen-shots of the TDK SEAT component selection tool, with
the 10$\mu$F C3225X5R1E106K capacitor selected. Details on
the figures are;
%
\begin{itemize}
%
\item
Figure~\ref{fig:tdk_seat_gui_Z_and_ESR} shows the impedance and ESR overlayed
on a single frequency plot. 

This plot was generated by;
\begin{itemize}
\item Selecting {\em Impedance} tab at the top of the GUI.
\item Plot impedance by clicking on the orange $|Z|$ button.
\item Plot ESR by clicking on the orange $R$ button.
\item Using the {\em Graph Setting} tab on the right-side of the figure,
change the x-axis of the plot from MHz to kHz.
\end{itemize}
%
\item Figure~\ref{fig:tdk_seat_gui_DC_bias} shows the capacitance value change
with DC bias.

This plot was generated by;
\begin{itemize}
\item Selecting {\em DC-Bias/Temp./Osc. Chara.} tab at the top of the GUI.
\item Under the {\em DC-Bias Characteristic} field at the top of the GUI $\dots$
\item There are two bullet options; {\em Value} and {\em Change}. 
Select {\em Value} to plot the change in capacitance in $\mu$F
(the {\em Change} option plots the change in percent). 
\item To the right of the bullets is a scroll menu, scroll the menu
until the orange $C_S$ (series capacitance) button is shown. 
\item Click the $C_S$ button.
\item Using the {\em Graph Setting} tab on the right-side of the figure,
change the y-axis of the plot from log to linear.
\end{itemize}
%
\item Figure~\ref{fig:tdk_seat_gui_power} shows the power handling
of the capacitor.

This plot was generated by;
\begin{itemize}
\item Selecting {\em Simulation/Tool} tab at the top of the GUI.
\item Clicking on the {\em Temp. Rise} button. This causes the 
{\em Temperature Rise Simulation} GUI to appear in the main window.
\item On the {\em Temperature Rise Simulation} GUI select the {\em Capacitor}
tab.
\item Under the {\em Sine Waveform} tab, select {\em Current Source}
and enter a {\em Frequency} of 500kHz. 
\item Click the {\em Simulate} button.
\item Move the mouse over the plot to show the current at a temperature
rise of 20$^\circ$C. This is the maximum current handling of the capacitor.
\end{itemize}
%
\end{itemize}
%
Each of the characteristic curves can be saved to .bmp, .jpg, or .png.
Figures~\ref{fig:tdk_Z_and_ESR} through~\ref{fig:tdk_DC_bias} show the
curves for the 10$\mu$ C3225X5R1E106K capacitor\footnote{
The ESL display in Figure~\ref{fig:tdk_ESL} has a {\em bug}; 
the data is offset along the x-axis by 1000 times.}.
%
The 
\href{http://www.tdk.com/pdf/etutorial_all.pdf}{SEAT Tutorial}
describes how to use the extensive features of the tool.

\clearpage
Table~\ref{tab:tdk_caps} shows a selection of TDK
ceramic capacitor part numbers obtained 
from~\href{http://www.digikey.com}{Digikey}, with
characteristics provided by the SEAT tool. 
%
Not all parts on Digikey have SEAT characterization data
(or complete data), eg., the table entry for C3225X7R1H105K
is missing the change in capacitance and the power parameters.
%
The thermal resistance, $\theta_{\rm JA}$, was obtained from the
default value given on the {\em Temperature Rise Simulation} GUI
(Figure~\ref{fig:tdk_seat_gui_power}) for each part. 
Table~\ref{tab:tdk_caps} shows that the thermal resistance is a 
constant value for each package size.
Power handling data can also be found on the TDK web site;
\href{http://www.tdk.com/tvcl_ircd.php}{Capacitors - Impedance and Ripple Current Data}.

% -----------------------------------------------------------------
% TDK SEAT GUI
% -----------------------------------------------------------------
%
\begin{landscape}
\begin{figure}[p]
  \begin{center}
    \includegraphics[width=210mm]{figures/tdk_seat_gui_Z_and_ESR.png}\\
  \end{center}
  \caption{TDK SEAT component selection tool; capacitor impedance and DC resistance.}
  \label{fig:tdk_seat_gui_Z_and_ESR}
\end{figure}
\end{landscape}

\begin{landscape}
\begin{figure}[p]
  \begin{center}
    \includegraphics[width=210mm]{figures/tdk_seat_gui_DC_bias.png}\\
  \end{center}
  \caption{TDK SEAT component selection tool; capacitor capacitance change with DC bias.}
  \label{fig:tdk_seat_gui_DC_bias}
\end{figure}
\end{landscape}

\begin{landscape}
\begin{figure}[p]
  \begin{center}
    \includegraphics[width=210mm]{figures/tdk_seat_gui_power.png}\\
  \end{center}
  \caption{TDK SEAT component selection tool; capacitor power handling.}
  \label{fig:tdk_seat_gui_power}
\end{figure}
\end{landscape}

% -----------------------------------------------------------------
% 10uF C3225X5R1E106K plots
% -----------------------------------------------------------------
%
% These were saved via right-clicking on the image and saving as
% .png, however, the .png files cause pdflatex to crash. The
% files were converted to .bmp and back to .png using
% png2bmp and bmp2png. Next time just save .bmp and convert
% to .png.
%
\begin{figure}[p]
  \begin{center}
    \includegraphics[width=0.8\textwidth]{figures/tdk_C3225X5R1E106K_Z_and_ESR.png}
  \end{center}
  \caption{TDK 10$\mu$F C3225X5R1E106K capacitor impedance and ESR vs frequency.}
  \label{fig:tdk_Z_and_ESR}
\end{figure}

\begin{figure}[p]
  \begin{center}
    \includegraphics[width=0.8\textwidth]{figures/tdk_C3225X5R1E106K_C.png}
  \end{center}
  \caption{TDK 10$\mu$F C3225X5R1E106K capacitor capacitance vs frequency.}
  \label{fig:tdk_C}
\end{figure}

\begin{figure}[p]
  \begin{center}
    \includegraphics[width=0.8\textwidth]{figures/tdk_C3225X5R1E106K_ESL.png}
  \end{center}
  \caption{TDK 10$\mu$F C3225X5R1E106K capacitor capacitor ESL vs frequency.}
  \label{fig:tdk_ESL}
\end{figure}

\begin{figure}[p]
  \begin{center}
    \includegraphics[width=0.8\textwidth]{figures/tdk_C3225X5R1E106K_power.png}
  \end{center}
  \caption{TDK 10$\mu$F C3225X5R1E106K capacitor change in temperature with RMS current.}
  \label{fig:tdk_power}
\end{figure}

\begin{figure}[p]
  \begin{center}
    \includegraphics[width=0.8\textwidth]{figures/tdk_C3225X5R1E106K_DC_bias_C.png}\\
    (a) change in capacitance\\
    $\quad$\\
    \includegraphics[width=0.8\textwidth]{figures/tdk_C3225X5R1E106K_DC_bias_dC.png}\\
    (b) relative change\\
  \end{center}
  \caption{TDK 10$\mu$F C3225X5R1E106K capacitor capacitance change with applied DC bias.}
  \label{fig:tdk_DC_bias}
\end{figure}

% -----------------------------------------------------------------
% TDK part numbers
% -----------------------------------------------------------------

\begin{landscape}
\begin{table}[p]
\small
\caption{Example TDK ceramic capacitor characteristics (@500kHz).}
\label{tab:tdk_caps}
\begin{center}
\begin{tabular}{|c|c|c|c|c|c||c|c|c|c||c||c|}
\hline
Part   & Cap. & Voltage & Tol. & Dielectric & Package &
\multicolumn{4}{c||}{DC Bias of 0V}  &
DC Bias of 50\% & Thermal\\
%
\cline{7-11}
Number &   &     & Rating     &            &         &
Cap. & ESR & ESL & Current    &
Cap. (change) & ($\theta_{\rm JA}$)\\
%
       & ($\mu$F) & (V) & (\%) & & & 
($\mu$F) & (m$\Omega$) & (pH) & (A$_{\rm RMS}$) &
($\mu$F) (\%) &($^\circ$C/W)\\
\hline\hline
&&&&&&&&&&&\\
C1608X5R1E105K &   1 &  25 & $\pm10$ & X5R & 0603 &  0.91 &  7.5 &  720 & 1.9 & 0.61 (-39.0) & 740 \\
C1608X5R1C475K & 4.7 &  16 & $\pm10$ & X5R & 0603 &   4.0 &  4.3 &  740 & 2.5 & 0.88 (-81.3) & 740 \\
C1608X5R1A106K &  10 &  10 & $\pm10$ & X5R & 0603 &   8.0 &  4.8 &  440 & 0.4 &  2.4 (-76.3) & 740 \\
&&&&&&&&&&&\\
C2012X5R1E105K &   1 &  25 & $\pm10$ & X5R & 0805 &  0.94 &  8.1 &  740 & 2.3 & 0.74 (-25.6) & 470 \\
C2012X5R1H475K & 4.7 &  50 & $\pm10$ & X5R & 0805 &   4.6 &  2.5 &  650 & 4.1 & 0.96 (-79.6) & 470 \\
C2012X5R1E106K &  10 &  25 & $\pm10$ & X5R & 0805 &  10.0 &  9.7 &  670 & 6.5 &  3.1 (-69.4) & 470 \\
C2012X5R1A476M &  47 &  10 & $\pm20$ & X5R & 0805 &  47.0 &  2.4 &  620 & 4.2 & 14.1 (-70.0) & 470 \\
&&&&&&&&&&&\\
C3216X5R1H105K &   1 & 100 & $\pm10$ & X5R & 1206 &  0.90 & 12.2 & 1100 & 2.1 & 0.46 (-53.9) & 380 \\
C3216X5R1E106K &  10 &  25 & $\pm10$ & X5R & 1206 &  10.0 &  2.2 &  890 & 4.9 &  5.8 (-42.0) & 380 \\
C3216X5R1E476M &  47 &  25 & $\pm20$ & X5R & 1206 &  47.0 &  1.6 &  930 & 5.8 &  5.2 (-89.0) & 380 \\
C3216X5R1A107M & 100 &  10 & $\pm20$ & X5R & 1206 & 100.0 &  1.7 &  910 & 5.5 & 26.6 (-73.4) & 380 \\
&&&&&&&&&&&\\
C3225X7R1H105K &   1 &  50 & $\pm10$ & X7R & 1210 &  0.97 & 7.5 & 1060 &     &              & \\
C3225X7R1H335K & 3.3 &  50 & $\pm10$ & X7R & 1210 &   3.3 & 4.1 & 1040 & 3.7 &  1.8 (-44.4) & 370 \\
C3225X5R1E106K &  10 &  25 & $\pm10$ & X5R & 1210 &  10.8 & 2.4 & 1000 & 4.2 &  7.6 (-24.4) & 370\\
C3225X5R0J107M & 100 & 6.3 & $\pm10$ & X7R & 1210 & 100.0 & 2.0 &  960 & 5.2 & 55.7 (-44.3) & 370 \\
&&&&&&&&&&&\\
\hline
\end{tabular}
\end{center}
\end{table}
\end{landscape}

\clearpage
% -----------------------------------------------------------------
\subsection{Discussion}
% -----------------------------------------------------------------

A comparison of the tables of capacitor parameters generated for
the different manufacturers provides the following generalizations;
%
\begin{itemize}
%
\item Package thermal resistance.

The estimates for the package thermal resistance of the Kemet capacitors
(Table~\ref{tab:kemet_caps}) are significantly lower than those
of the Murata capacitors (Table~\ref{tab:murata_caps}), the
Taiyo Yuden capacitors (Table~\ref{tab:taiyo_yuden_caps}), and
the TDK capacitors (Table~\ref{tab:tdk_caps}).

Its not clear whether the Kemet capacitors are really more thermally 
conductive than the Murata, Taiyo Yuden and TDK parts, or the temperature
rise was simply measured using a different method.

The TDK thermal resistance values should be used when estimating
power dissipation (since the thermal resistance was explicitly provided,
and they are of similar magnitude to the Murata and Taiyo Yuden values).

\item Equivalent Series Inductance (ESL).

The Kemet capacitors (Table~\ref{tab:kemet_caps}) have the highest
estimates, while the Murata capacitors (Table~\ref{tab:murata_caps})
have the lowest estimates.
%
The variation in ESL is likely due to Murata de-embedded the inductance
of the surface-mount pads and solder filets. The lower valued
estimates should be considered conservative.
\end{itemize}
%

\clearpage
% =================================================================
\section{Power Supply Design Examples}
% =================================================================

% -----------------------------------------------------------------
\subsection{Input Capacitors}
% -----------------------------------------------------------------

{\bf TODO:}
\begin{itemize}
\item Single-phase and dual-phase LTspice power-supply, eg.,
LTC3855 or LTC3880 0.95V@20A and 40A.
\item Show the input waveforms for an ideal capacitor,
capacitor plus ESR, and capacitor plus ESR plus ESL.
\item Show the input waveforms for some of the vendor
capacitor models.
\item Show the current through the high-side MOSFET, and
then show how it splits between the input power source and
the input decoupling depending on the impedance of the
power source, eg., with a 1-Ohm series resistor, the switching
current comes mostly from the input caps, but there is a large
input voltage drop over the resistor (but that is ok, since this
is just a simulation to estimate the RMS current in the caps).
Measure the input cap RMS current for various input source
resistances.
\end{itemize}

% -----------------------------------------------------------------
\subsection{Output Capacitors}
% -----------------------------------------------------------------

{\bf TODO:}
\begin{itemize}
\item Load transient increase/decrease load dip/peaking formula.
\item Sanyo POSCAPs with ceramics in parallel.
\end{itemize}

\clearpage
% -----------------------------------------------------------------
% Do the bibliography
% -----------------------------------------------------------------
%Note, you can't have spaces in the list of bibliography files
%
\bibliography{refs}
\bibliographystyle{plain}

% -----------------------------------------------------------------
\end{document}











